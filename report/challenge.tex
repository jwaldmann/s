\documentclass[12pt]{article}
\usepackage[text={6in,9in},centering]{geometry}
\usepackage{amsfonts}
\usepackage{amsmath}
\usepackage{url}
\usepackage{hyperref}
\usepackage[utf8]{inputenc}
\newcommand{\NN}{\mathbb{N}}
\newcommand{\CL}{\operatorname{\textbf{CL}}}
\newcommand{\Term}{\operatorname{Term}}
\newcommand{\SN}{\operatorname{SN}}
\newcommand{\Lang}{\operatorname{Lang}}

\title{Challenges for Termination Provers \\
  (Properties of the S Combinator) \\
  DRAFT
}
\author{Johannes Waldmann}

\begin{document}
\maketitle

\section{Introduction}

We propose several benchmarks
that can be used to evaluate power and progress
of software that analyzes termination 
and derivational complexity of term rewriting.

In particular, we are interested in
\begin{itemize}
\item a proof of termination
\item a proof of quadratic innermost derivational complexity
\item a proof of context-sensitive termination
\item a proof of relative top-termination
\end{itemize}
for systems with a large number of rules (approx. 2000 rules)
or a medium-sized signature (approx. 40 symbols).

These benchmarks have a common source: 
they originate from the study of the combinator $S$
\cite{DBLP:journals/iandc/Waldmann00}.
This is the rewriting system 
\[ \{ A(A(A(S,x),y),z)\to A(A(x,z),A(y,z)) \} \]
over signature $\{S/0, A/2\}$.
Using methods from (cite: local termination),
\cite{DBLP:journals/corr/abs-1006-4955},
we obtain the challenge systems by semantic labelling
\cite{DBLP:journals/fuin/Zantema95}.
We use two models: the underlying tree automaton
describes the normalizing terms, 
and the (conjectured) head-normalizing terms, respectively.


\section{Termination}

\emph{Motivation:}
We have deterministic (incomplete) tree automaton $A$
(on 38 states)
and we want to verify that it accepts exactly 
the normalizing $S$-terms.

We need to check that 
\begin{enumerate}
\item \label{it:fa} 
  $A$ is closed w.r.t. rewriting,
\item \label{it:fb} 
  $\CL(S)$ is locally terminating on $\Lang(A)$,
\item \label{it:fc} 
  each tree in $\Term(\Sigma)\setminus\Lang(A)$
  contains a redex.
\end{enumerate}
It is easy to do check items \ref{it:fa} and \ref{it:fc}
by inspection. The challenge is item \ref{it:fb}.
For that, we construct a TRS $F$ 
by labelling rules of $\CL(S)$ with states of $A$, obtaining
\url{https://github.com/jwaldmann/s/blob/master/full.trs}

\emph{Challenge:} Prove termination of $F$. 
(has approx. 2000 rules)


\emph{Status:} this can be solved at least by 
\texttt{matchbox2014},
using DP transformation, EDG decomposition,
and matrix interpretations (dimension at most 2),
producing a termination proof that can be verified by 
\texttt{CeTA}.

\section{Innermost Derivational Complexity}

\emph{Motivation}: We have conjecture (1998) 
that $\CL(S)$ has quadratic innermost derivational complexity
\cite{DBLP:conf/flops/AvanziniM08}
on the normalizing terms.

\emph{Challenge}: prove quadratic
innnermost derivational complexity for TRS $F$
from previous challenge.

\emph{Status}: (open)

\section{Termination of Context-Sensitive Rewriting}

\emph{Motivation}:
We have a deterministic complete tree automaton $B$
(on 35 states)
and we want to verify that it accepts 
exactly the head-normalizing $S$-terms.

We need to check that 
\begin{enumerate}
\item \label{it:ha} 
  $B$ is closed w.r.t. head rewriting,
\item \label{it:hb} 
  $\CL(S)$ is locally terminating on $\Lang(B)$
  for head reductions
\item \label{it:hc} 
  each tree in $\Term(\Sigma)\setminus\Lang(B)$
  contains a head redex.
\end{enumerate}
It is again easy to do check 
items \ref{it:ha} and \ref{it:hc}
by inspection. The challenge is item \ref{it:hb}.
For that, we construct a TRS $H$ by labelling rules
of $\CL(S)$ with states of $B$, obtaining
\url{https://github.com/jwaldmann/s/blob/master/head.trs}

Since the focus is termination of head reductions,
we need to restrict the rewrite relation.
This can be done with the model of
\emph{context-sensitive rewriting}, 
\cite{DBLP:journals/entcs/AlarconGIL07},
and we declare
only the left argument position of (labelled)
application symbols as replacing position.

\emph{Challenge}: prove context-sensitive termination
of $H$ (8000 rules).

\emph{Status}: (open)

\section{Relative Top-Termination (Small)}

\emph{Motivation}:
We proved (pen-and-paper) that $\CL(S)$ is top-terminating.
We want this proof obtained automatically, and verified.

\emph{Challenge}: prove relative termination
\[ \SN( \{A^\#(A(A(S,x),y),z)\to A^\#(A(x,z),A(y,z)) \},
\{ A(A(A(S,x),y),z)\to A(A(x,z),A(y,z)) \} ) 
\]
\emph{Remark}: since the $A^\#$ symbol only occurs
at the very top, this is actually 
a dependency pairs problem
\cite{DBLP:journals/tcs/ArtsG00}
and some of the solvers might use this fact.

\emph{Status}: (open)

\section{Relative Top-Termination}

\emph{Motivation} Referring to the previous problem,
it may be good to help the solver
by semantic labelling, e.g., using the automaton $B$.

\emph{Challenge}: prove relative termination
of the labelled system.

\emph{Status}: (open)

\section{Discussion}

Hardness of these challenges is primarily 
in the sheer size of these systems (thousands of rules).
We believe this is a good test case for other
applications of automated termination and complexity
analyzers. Application problems might lead to
termination problems that are quite different
from the ``crafted'' problems 
that constitute the bulk of TPDB today.
In particular, they may be large, but could be solved 
by a combination of easy steps --- but a large number 
of them (as in matchboxes' answer to the first challenge).
We note that other solver competitions (e.g., SAT) 
have a separation in crafted, random,
and application problems---so these communities
also think that these categories pose different challenges
to solvers, and each is worth pursuing.

\bibliographystyle{alpha}
\bibliography{challenge}


\end{document}

